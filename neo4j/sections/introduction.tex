\documentclass[../main.tex]{subfiles}

\graphicspath{{\subfix{../images/}}}
\begin{document}

% % Example Cypher query in lstlisting
% \begin{lstlisting}[language=cypher, caption={Example Cypher Query}]
% /* Find friends of Alice older than 30 */
% MATCH (a:Person)-[:KNOWS]->(b:Person)
% WHERE a.name = "Alice" AND b.age > 30
% RETURN b.name AS friend, b.age AS age
% ORDER BY b.age DESC
% LIMIT 10;
% 
% // Create a new person named Bob
% CREATE (b:Person {name: "Bob", age: 25});
% 
% // Set Bob's location
% MATCH (b:Person {name: "Bob"})
% SET b.location = "New York";
% 
% // Delete relationships for Bob
% MATCH (b:Person {name: "Bob"})-[r]-()
% DELETE r
% \end{lstlisting}

\chapter{Introduction}
\section{Setting up Neo4j with docker}
Using official docker setup \href{https://hub.docker.com/_/neo4j}{\color{blue}{link}} for neo4j.
\begin{lstlisting}[language=bash, caption={get neo4j image}]
docker pull neo4j
\end{lstlisting}
\begin{lstlisting}[language=bash, caption={Starting neo4j container}]
docker run \
		--publish=7474:7474 --publish=7687:7687 \
		--volume=$HOME/neo4j/data:/data \
		neo4j
\end{lstlisting}

\section{Creating Nodes}
\begin{lstlisting}[language=cypher, caption={Create Node}]
CREATE( var: label{key1:value1, key2:value2, key3:value3,...,keyn:valuen})
\end{lstlisting}
\begin{itemize}
	\item{Properties of nodes present between \{ and \}}
	\item{ 
			Node present between \( and \)
			\begin{enumerate}
				\item{var is the variable name such as John}
				\item{label is the entity type such as Customer}
				\item{key:value are the attributes of the properties such as  \lstinline{contact_num: 1234567890}}
			\end{enumerate}
		}
\end{itemize}


\printglossaries
\end{document}
