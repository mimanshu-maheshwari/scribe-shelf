\documentclass[../main.tex]{subfiles}

\graphicspath{{\subfix{../images/}}}
\begin{document}

% % Example Cypher query in lstlisting
% \begin{lstlisting}[language=cypher, caption={Example Cypher Query}]
% /* Find friends of Alice older than 30 */
% MATCH (a:Person)-[:KNOWS]->(b:Person)
% WHERE a.name = "Alice" AND b.age > 30
% RETURN b.name AS friend, b.age AS age
% ORDER BY b.age DESC
% LIMIT 10;
% 
% // Create a new person named Bob
% CREATE (b:Person {name: "Bob", age: 25});
% 
% // Set Bob's location
% MATCH (b:Person {name: "Bob"})
% SET b.location = "New York";
% 
% // Delete relationships for Bob
% MATCH (b:Person {name: "Bob"})-[r]-()
% DELETE r
% \end{lstlisting}

\section{Setting up Neo4j with docker}
Using official docker setup \href{https://hub.docker.com/_/neo4j}{\color{blue}{link}} for neo4j.
\begin{lstlisting}[language=bash, caption={get neo4j image}]
docker pull neo4j
\end{lstlisting}
\begin{lstlisting}[language=bash, caption={Starting neo4j container}]
docker run \
		--publish=7474:7474 --publish=7687:7687 \
		--volume=$HOME/neo4j/data:/data \
		neo4j
\end{lstlisting}

\section{Creating Nodes}
\begin{lstlisting}[language=cypher, caption={Create Node}]
CREATE( var: label{key1:value1, key2:value2, key3:value3,...,keyn:valuen})
\end{lstlisting}
\begin{itemize}
	\item{Properties of nodes present between \{ and \}}
	\item{ 
			Node present between \( and \)
			\begin{enumerate}
				\item{var is the variable name such as John}
				\item{label is the entity type such as Customer}
				\item{key:value are the attributes of the properties such as  \lstinline{contact_num: 1234567890}}
			\end{enumerate}
		}
\end{itemize}
Example: 
\begin{lstlisting}[language=cypher, caption={Create Node Example}]
CREATE (b:Bank{name:"Oziku"})
\end{lstlisting}


\section{Creating Relationships}
\begin{lstlisting}[language=cypher, caption={Create Relationships}]
CREATE (node1) -[var:Rel_type{key1:value1, key2:value2, ...keyn:valuen}] -> (node2)
\end{lstlisting}
Example: 
\begin{lstlisting}[language=cypher, caption={Create Relation Example}]
CREATE (BG:Customer { cust_id: 675489, cust_name: "BlackGreyTechnologies", contact_num: 1298764592 })-
[:Owns]->(AcBG:Account { acc_num: 3498761, cust_id: 675489, type: "checking", balance:958990 })
\end{lstlisting}
Nodes, relationships can be created with/without label, properties
Nodes or relationships can have single or multiple labels
\section{Defining a constraint}
Create a unique constraint on \lstinline{cust_id} property of the Customer node as shown below:
\begin{lstlisting}[language=cypher, caption={constraint}]
CREATE CONSTRAINT ON (c:Customer) ASSERT c.cust_id IS UNIQUE
\end{lstlisting}
\section{MATCH}
Is used to search for a pattern.
To return all the nodes created:
\begin{lstlisting}[language=cypher, caption={Match syntax}]
MATCH (n) RETURN n
\end{lstlisting}

\begin{lstlisting}[language=cypher, caption={Match example}]
MATCH (n:Bank) 
RETURN n 
\end{lstlisting}

\begin{lstlisting}[language=cypher, caption={Match example}]
MATCH (Customer { cust_name: 'BlackGreyTechnologies' })--(Account)
RETURN Account.acc_num, Account.balance
\end{lstlisting}

\begin{lstlisting}[language=cypher, caption={Match example}]
MATCH (city:City {name:"Raleigh"})
MERGE (state:State{name:"North Carolina"})
MERGE (city)-[:LOCATED_IN]->(state)
RETURN city,state
\end{lstlisting}

\section{Optional Match}
\begin{lstlisting}[language=cypher]
MATCH (c:Customer { cust_id: 675489 })
OPTIONAL MATCH (c)-[r:having] -()
RETURN c.cust_name, r.name
\end{lstlisting}

\section{WHERE}
\begin{lstlisting}[language=cypher]
MATCH (c:Customer), (a:Account)
WHERE a.type='checking'
RETURN c.cust_name,a.acc_num,a.balance
\end{lstlisting}

\section{Merge}
\begin{lstlisting}[language=cypher]
MERGE (c:Customer { cust_name: 'Charlie Sheen' })
RETURN c
\end{lstlisting}


\section{UNION, UNION ALL}
Union combines the results of multiple queries and removes duplicates whereas Union All performs the same operation but retains duplicates.
\begin{lstlisting}[language=cypher]
MATCH (c:Customer), (a:Account)
WHERE a.type='checking'
RETURN c.cust_name,a.acc_num,a.balance
UNION
MATCH (c:Customer { cust_name: 'BlackGreyTechnologies' })--(a:Account)
RETURN c.cust_name, a.acc_num, a.balance
\end{lstlisting}
NOTE: All sub queries must have the same column names.

\section{ORDER BY, SKIP, LIMIT, DISTINCT}
ORDER BY - specifies how the output of RETURN or WITH should be sorted.
SKIP - defines from which record to start including the records in the output.
LIMIT - constraints the number of output records.
DISTINCT - retrieves only unique rows.
The below query sorts the Customer node in the ascending order of its customer names, skips the first record and limits the output to only one record:
\begin{lstlisting}[language=cypher]
ORDER BY, SKIP, LIMIT, DISTINCT
\end{lstlisting}
To retrieve the unique relationships present in the database:
\begin{lstlisting}[language=cypher]
MATCH (n)-[r]-()
RETURN distinct type(r)
\end{lstlisting}

\section{SET}
It is used to update node labels and properties of nodes and relationships.
For example, adding new properties such as email and country to the existing Customer BlackGreyTechnologies:

\begin{lstlisting}[language=cypher, caption={Match example}]
MATCH (c:Customer {cust_name: 'BlackGreyTechnologies'})  
SET c.email= 'BGT@blackgrey.com', c.country = 'France' 
RETURN c
\end{lstlisting}


\section{FOREACH}
It updates data within a list which can be components of a path* or result of an aggregation
Path is a directed sequence of nodes and relationships.
Assume you want to track funds transfer between two accounts suspected to be laundering funds illegally between intermediary accounts.
\begin{lstlisting}[language=cypher]
CREATE p = (:Account {acc_num:65178})-[:Funds_transfer]->(:Account
{acc_num:98567})-[:Fundstransfer]->(:Account{acc_num:46378})-[:Fundstransfer]->(:Account
{acc_num:95648})-[:Fundstransfer]->(:Account{acc_num:46897})
RETURN p

MATCH p = (:Account{acc_num:65178})-[*]->(:Account{acc_num:46897}) 
FOREACH (n IN nodes(p)| SET n.marked = "flaggedFraud")

\end{lstlisting}
Note: In the above code, [*] is used to define any number of intermediary relationships \lstinline{acc_num=65178 to acc_num=46897}
In the output shown below, you can observe that a new property \lstinline{marked:"flaggedFraud"} has been added:

\section{REMOVE}
Is used to remove labels and properties of nodes and relationships.

To remove the email property from Customer having \lstinline{cust_name=BlackGreyTechnologies}:
\begin{lstlisting}[language=cypher]
MATCH (c:Customer {cust_name: 'BlackGreyTechnologies'}) 
REMOVE c.email
RETURN c
\end{lstlisting}

\section{DELETE}
Is used to delete nodes, relationships or paths.
Node cannot be deleted without deleting its associated relationships. Either delete relationships explicitly or use DETACH DELETE that is discussed next.
\begin{lstlisting}[language=cypher]
MATCH (n { acc_num: 65178 })-[r:Funds_transfer]->()
DELETE r
\end{lstlisting}

\section{DETACH DELETE}
Is used to delete nodes along with their relationships.

To delete Customer node with \lstinline{name:"BlackGreyTechnologies"} and all its associated links:

\begin{lstlisting}[language=cypher]
MATCH (a {cust_name: "BlackGreyTechnologies"}) 
DETACH DELETE a
\end{lstlisting}
Delete all the nodes and relationships from the database:
\begin{lstlisting}[language=cypher]
MATCH (n)
OPTIONAL MATCH (n)-[r]-()
DELETE n,r
\end{lstlisting}

\section{Bulk Load of Data}

Step 1: 
To load data from csv file, the below configuration property needs to be added to the neo4j.conf configuration file:
\lstinline{dbms.security.allow_csv_import_from_file_urls=true}

Step 2:
Create Bank node, Customer node and ensure that the customer ID is unique as discussed previously.

Step 3:
Load data from Customer.csv file to Customer node
\begin{lstlisting}[language=cypher]
load csv with headers from 'file:///C:/Users/Sahana_Basavaraja/Desktop/Customer.csv' as cust
merge (c:Customer{cust_id:toInteger(cust.cid),cust_name:cust.cname,contact_num:toInteger(cust.phnum)})
\end{lstlisting}
Note: By default, the row retrieved from the file is always string, for example cust. You need to explicitly convert to the appropriate datatypes if required, for example \lstinline{toInteger()} as shown above.
Establish the relationship between Customer and Bank as \lstinline{Customer_of}:
\begin{lstlisting}[language=cypher]
MATCH (c:Customer),(b:Bank) merge (c)-[b1:Customer_of]->(b)
MATCH (n) RETURN n
CREATE CONSTRAINT ON (a:Account) ASSERT a.acc_num IS UNIQUE

// load data present in account csv
load csv with headers from 'file:///C:/Users/Sahana_Basavaraja/Desktop/Account.csv' as acc
merge(a:Account{acc_num:toInteger(acc.acc_num)})
set 
a.cust_id=toInteger(acc.cid),
a.balance=toInteger(acc.balance)

\end{lstlisting}
Note: Use SET to map columns of the file to properties of the node as illustrated above.
Establish the relationship between Customer and Account as Owns:

\begin{lstlisting}[language=cypher]
load csv with headers from 'file:///C:/Users/Sahana_Basavaraja/Desktop/Account.csv' as acc
match(a:Account{acc_num:toInteger(acc.acc_num)}),(c:Customer{cust_id:toInteger(acc.cid)})
merge (a)<-[o:Owns]-(c)

\end{lstlisting}

\begin{lstlisting}[language=cypher]
load csv with headers from 'file:///C:/Users/Sahana_Basavaraja/Desktop/txn.csv' as txn
match (a:Account), (b:Account)
where a.acc_num=toInteger(txn.from_acc) AND b.acc_num=toInteger(txn.to_acc) AND toInteger(a.balance)>=toInteger(txn.amount_acc)
create (a)-[:Funds_transfer{txn_id:toInteger(txn.tid),from_acc:toInteger(txn.from_acc),to_acc:toInteger(txn.to_acc),amount:toInteger
(txn.amount_acc),txntime:txn.txntime}]->(b)
\end{lstlisting}

\section{Functions}
\subsection{toLower()}
\begin{lstlisting}[language=cypher]
MATCH (c:Customer) RETURN toLower(c.cust_name) LIMIT 1
\end{lstlisting}

\subsection{toUpper()}
\begin{lstlisting}[language=cypher]
MATCH (c:Customer) RETURN toUpper(c.cust_name) LIMIT 1
\end{lstlisting}

\subsection{substring()}
\begin{lstlisting}[language=cypher]
MATCH (c:Customer) RETURN SUBSTRING(c.cust_name,0,5) LIMIT 1
\end{lstlisting}

\subsection{replase()}
\begin{lstlisting}[language=cypher]
MATCH (c:Customer{cust_name:"BlackGreyTechnologies"}) RETURN replace("BlackGreyTechnologies","Grey","Gradient")
\end{lstlisting}

\subsection{reverse()}
\begin{lstlisting}[language=cypher]
MATCH (c:Customer) RETURN reverse(c.cust_name) LIMIT 1
\end{lstlisting}

\subsection{Aggregate function}
It is similar to GROUP BY clause in SQL. It takes multiple values as arguments, performs computation on it and returns the computed value.

\begin{itemize}
	\item{sum() - returns the sum of a set of values}
	\item{avg() - returns the average of a set of values}
	\item{count() - returns number of rows or values}
	\item{max() - returns maximum value in set of values}
	\item{min() - returns minimum value in set of values}
\end{itemize}
 

The frauds do not transfer amount in one single transaction.
Multiple transactions would take place in order to hide the transit of illegally obtained money.

The below query retrieves the details of the account number having highest \lstinline{total_amount} transferred in a series that is greater than 20K:
\begin{lstlisting}[language=cypher]
MATCH (a)-[r:Funds_transfer]->(b)
WITH r.to_acc AS account_number,
count(r.to_acc) as total_txn_to_acc,
sum(r.amount) as total_amount
WHERE total_amount >20000
RETURN account_number,total_amount,total_txn_to_acc
ORDER BY total_amount DESC
LIMIT 1
\end{lstlisting}

\subsection{List functions}
It returns a list of nodes and relationships in a path, labels, keys.

\begin{itemize}
	\item{keys() - returns a list of property names of a node, relationship or map}
	\item{labels() - returns a list of all the labels of a node}
	\item{nodes() - returns a list of all nodes in a path}
	\item{relationships() - returns a list of all relationships in a path}
\end{itemize}
To retrieve the list of distinct labels and its respective property keys:
\begin{lstlisting}[language=cypher]
MATCH (n)
RETURN labels(n), keys(n)
\end{lstlisting}

\section{User defined functions(UDF) in Neo4J}
Neo4J has many built-in functions. In order to extend the functionalities of Neo4J, you can create your own UDFs  in Java.
\subsection{UDFs in Neo4J}
\begin{itemize}
	\item{are read-only and always returns a single-value}
	\item{is annotated with \lstinline{@UserFunction}}
	\item{valid input types and output types are \lstinline{string, long, double, boolean, node, relationship, path, object, map<K(string),V>, list}}
	\item{is called as \lstinline{package-name.method-name}}
\end{itemize}
From the Account node, retrieve the average balance amount.
\begin{lstlisting}[language=java]
package com.infy.bda;
import java.util.List;
import org.neo4j.procedure.Description;
import org.neo4j.procedure.Name;
import org.neo4j.procedure.UserFunction;
public class AverageBalance {
    @UserFunction
    @Description("com.infy.bda.AverageBalance([0.5,1,2.3]) returns the average of the given list of values")
    public double avg(@Name("numbers") List<Number> list) {
        double avg = 0;
        for (Number number : list) {
            avg += number.doubleValue();
        }
        return (avg/(double)list.size());
    }
}
\end{lstlisting}

\subsection{Deploying UDF}
UDFs are packaged in a jar file. You have to copy the jar file into the plugins directory of your Neo4J server and restart.
\subsection{Calling a UDF}
Before calling the UDF, to see the list of UDFs deployed in your server, use the following command:
\begin{lstlisting}[language=cypher]
CALL dbms.functions()
\end{lstlisting}
User-defined functions are called in the same way as any other Cypher function. The function name must be fully qualified, that is, a function named avg defined in the package \lstinline{com.infy.bda} is called using: 
\begin{lstlisting}[language=cypher]
MATCH(a:Account)
RETURN com.infy.bda.avg(collect(a.balance))
\end{lstlisting}


\section{Optional Schema Indexes and constraints}
\subsection{Analyze query performance using EXPLAIN}
Query performance can be analysed using the EXPLAIN clause. It provides the execution plan for the Cypher query.
Consider the below query that retrieves the Customer node having \lstinline{cust_name:"George Clooney"}:
\begin{lstlisting}[language=cypher]
EXPLAIN
MATCH(c:Customer{cust_name:"George Clooney"})
RETURN c
\end{lstlisting}
Few properties generated from the output of EXPLAIN command are explained below:
\begin{itemize}
	\item NodeByLabelScan: Scanning the nodes based on a specific label
	\item Filter: Conditional expression
\end{itemize}
\subsection{Improve performance using indexes}
\subsubsection{Create Index:}
\begin{lstlisting}[language=cypher]
CREATE INDEX ON :Label(property) //Index on single property 
CREATE INDEX ON :Label(property1, property2, ...,propertyn) //Index on multiple properties
\end{lstlisting}
Example:
\begin{lstlisting}[language=cypher]
CREATE INDEX ON :Customer(cust_name)
\end{lstlisting}
\subsubsection{View Index:}
\begin{lstlisting}[language=cypher]
CALL db.indexes()
\end{lstlisting}

\subsubsection{Drop Index:}
\begin{lstlisting}[language=cypher]
DROP INDEX ON :Label(property)
DROP INDEX ON :Label(property1, property2, .....propertyn)
\end{lstlisting}
Example: 
\begin{lstlisting}[language=cypher]
DROP INDEX ON :Customer(cust_name)
\end{lstlisting}

\section{Working with constraints}
Data integrity in Neo4J can be enforced using constraints. Two types of constraints can be created:
\begin{itemize}
	\item Unique node property constraint
	\item Property and relationship existence constraint\footnote{\small{Available only in the Neo4J Enterprise Edition.}}
\end{itemize}
\subsection{Unique node property constraint}
It ensures that property values for all nodes corresponding to a particular label are unique.
\subsubsection{Creation of unique node constraint}
Earlier in the course, you have created a unique constraint on the \lstinline{cust_id} property of the Customer node as shown below:
\begin{lstlisting}[language=cypher]
CREATE CONSTRAINT ON (c:Customer) ASSERT c.cust_id IS UNIQUE
\end{lstlisting}
NOTE: 
\begin{itemize}
\item Nodes without the property used in the constraint will have no effect
\item Creation of unique property constraint will also create index on that property
\end{itemize}
\subsubsection{Drop unique constraint:}
Use DROP CONSTRAINT clause to remove a constraint from the database as shown below:
\begin{lstlisting}[language=cypher]
DROP CONSTRAINT ON (c:Customer) ASSERT c.cust_id IS UNIQUE
\end{lstlisting}
Note: Dropping the constraint will also remove the index created.

\section{Using Neo4J with Java}
Neo4J provides developer kits for various languages like Java, .NET, Python, PHP, Go, C and so on. Below code can be used to access Neo4J server from Java application using Neo4J's binary Bolt\footnote{Bolt protocol is a connection oriented network protocol used for client-server communication in database applications. By default, it runs on 7687 port.} protocol.
You can copy paste the following code in Eclipse and include \href{https://mvnrepository.com/artifact/org.neo4j/neo4j}{\color{blue}{neo4j}} and \href{https://mvnrepository.com/artifact/org.neo4j.driver/neo4j-java-driver}{\color{blue}{neo4j-java driver}} jars.
\begin{lstlisting}[language=java, caption={Neo4J with java}]
package com.infy;

import org.neo4j.driver.v1.AccessMode;
import org.neo4j.driver.v1.AuthTokens;
import org.neo4j.driver.v1.Driver;
import org.neo4j.driver.v1.GraphDatabase;
import org.neo4j.driver.v1.Session;
import org.neo4j.driver.v1.StatementResult;
import org.neo4j.driver.v1.Transaction;
import static org.neo4j.driver.v1.Values.parameters;
import java.util.ArrayList;
import java.util.List;

public class NeoJavaDemo implements AutoCloseable {
    private Driver driver;

    public NeoJavaDemo( String uri, String user, String password ) {
        this.driver = GraphDatabase.driver( uri, AuthTokens.basic( user, password ) );
    }

    @Override
    public void close() throws Exception {
        this.driver.close();
    }
    
    // Create a Bank node
    private StatementResult addBank( final Transaction tx, final String name ) {
        return tx.run( "CREATE (:Bank {name: $name})", parameters( "name", name ) );
    }
    
    // Create a Customer node
    private StatementResult addCustomer( final Transaction tx, final String cust_name ) {
        return tx.run( "CREATE (:Customer {cust_name: $cust_name})", parameters( "cust_name", cust_name ) );
    }
    
    // Create a customer_of relationship to a pre-existing bank node.
    // This relies on the customer first having been created.
    private StatementResult cust( final Transaction tx, final String customer, final String bank ) {
     return tx.run( "MATCH (c:Customer {cust_name: $cust_name}) " +
                     "MATCH (b:Bank {name: $name}) " +
                     "CREATE (c)-[:Customer_of]->(b)",
             parameters( "cust_name", customer, "name", bank ) );
    }
    
    public void addCustomerAndBank() {
        // To collect the session bookmarks
        List<String> savedBookmarks = new ArrayList<>();
        // Create the first customer and bank relationship.
        try ( Session session1 = this.driver.session( AccessMode.WRITE ) ) {
            session1.writeTransaction( tx -> addBank( tx, "Oziku" ) );
            session1.writeTransaction( tx -> addCustomer( tx, "George" ) );
            session1.writeTransaction( tx -> cust( tx, "George", "Oziku" ) );
            savedBookmarks.add( session1.lastBookmark() );
        }
    }
    
    public static void main( String[] args ) throws Exception {
        try ( NeoJavaDemo g = new NeoJavaDemo( "bolt://localhost:7687", "neo4j", "Infy@123#" ) ) {
            g.addCustomerAndBank();
        }
    }
}
\end{lstlisting}

\begin{itemize}
	\item The above Neo4J client connects to the Neo4J server running on localhost with \lstinline{neo4j} and \lstinline{Infy@123#} as username and password respectively.
	\item To connect to a remote Neo4J server, replace localhost with the server's hostname and provide relevant username and password.
\end{itemize}

For more details view \href{https://neo4j.com/docs/driver-manual/1.7/}{\color{blue}{Neo4J Drivers}}.






\printglossaries
\end{document}
