\documentclass[../main.tex]{subfiles}

% \begin{lstlisting}[language=GraphQL, caption={Employee fetch id and name only using GraphQL query}, label={lst:fetch-employee-id-first-name}]
% {
%   employees {
%     id
%     firstName
%   }
% }
% \end{lstlisting}
% Listing  \ref{lst:fetch-employee-id-first-name} is a basic example of usage.

\graphicspath{{\subfix{../images/}}}
\begin{document}
\chapter{OCI}
% 01. What is OCI?
\section{What is OCI?}
\acrfull{oci} is a cloud computing service offered by Oracle that provides \acrfull{iaas}, \acrfull{paas}, \acrfull{saas}, and \acrfull{daas} solutions.
It is designed to support high-performance computing, enterprise workloads, and cloud-native applications.
\acrshort{oci} provides services such as compute, storage, networking, databases, security, and \acrshort{ai}/\acrshort{ml} tools.

% 02. What is region in OCI?
\section{What is region in OCI?}
A region in \acrshort{oci} is a geographically distinct area where \acrshort{oci} resources are deployed.
Each region consists of multiple Availability Domains (ADs), ensuring high availability and disaster recovery.
Examples of \acrshort{oci} regions include "us-ashburn-1" (US East), "uk-london-1" (UK), "ap-mumbai-1" (India), etc.

% 03. What is realm in OCI?
\section{What is realm in OCI?}
A realm in OCI is a collection of OCI regions that share the same identity namespace and authentication policies.
Each OCI realm is isolated from others.
\begin{itemize}
  \item {Example: The commercial realm contains most OCI public regions, while the government realm includes regions meant for government customers.}
  \item {Common realms include:
      \begin{itemize}
        \item oc1 $\rightarrow$ Commercial cloud
        \item oc2 $\rightarrow$ Government cloud
        \item oc3 $\rightarrow$ Dedicated Oracle Cloud
      \end{itemize}
    }
\end{itemize}

% 04. What is physical domain and fault domain in OCI?
\section{What is a physical domain and fault domain in OCI?}
\begin{itemize}
  \item {\acrfull{ad}: A physically independent data center within an OCI region. 
      It has its own power, cooling, and network. Some OCI regions have multiple ADs.
      \acrshort{ad} are one or multiple data center that are localized in each region.
      They have their own network, cooling, networking, etc.
    }
  \item {\acrfull{fd}: A logical grouping of resources within an AD to protect against hardware failures.
    If a fault domain fails, only resources in that domain are affected.}
\end{itemize}
Example: If a region has 3 \acrlong{ad}, each AD has multiple \acrlong{fd} to distribute workloads and avoid single points of failure.

% 05. What is tenancy in OCI?
\section{What is tenancy in OCI?}
A tenancy in OCI is a root-level account that represents an organization’s cloud environment.
It contains compartments, users, groups, and policies that control access to resources.
Example: If a company has multiple departments, each department can have its own compartment within a single tenancy.

% 06. What is compartment in OCI?
\section{What is compartment in OCI?}
A compartment is a logical container for OCI resources such as compute instances, storage, and databases.
It allows for fine-grained access control using OCI \acrfull{iam} policies.
Example: A company might create separate compartments for Development, Testing, and Production environments.

% 07. What is group and dynamic group in OCI?
\section{What is a group and dynamic group in OCI?}
\begin{itemize}
  \item Group: A collection of IAM users that share the same access policies.
  \item Dynamic Group: A special type of group that includes compute instances and other cloud resources instead of users.
\end{itemize}
Example: If you want all compute instances in a compartment to access Object Storage, you create a dynamic group and grant permissions via IAM policies.

% 08. How is auth done in OCI?
\section{How is authentication done in OCI?}
OCI supports multiple authentication methods:
\begin{itemize}
  \item{Username \& Password: Standard login via OCI Console}
  \item{\acrfull{api} Key Authentication: Uses public-private key pairs for programmatic access}
  \item{Instance Principal Authentication: Allows compute instances to access resources without credentials}
  \item{Identity Federation: Integrates with Okta, \acrfull{maad}, and other \acrfull{IdP}s for \acrfull{sso}}
  \item{Resource Principal Authentication: Allows OCI services to authenticate with other OCI services}
\end{itemize}

% 09. Understanding IAM in OCI?
\section{Understanding IAM in OCI?}
\acrshort{oci} \acrfull{iam} controls who can access what within OCI. Key \acrshort{iam} concepts include:
\begin{itemize}
  \item Users $\rightarrow$ Individual accounts
  \item Groups $\rightarrow$ Collections of users with shared permissions
  \item Policies $\rightarrow$ Define permissions using a human-readable language
  \item Compartments $\rightarrow$ Logical containers for resources
  \item Dynamic Groups $\rightarrow$ Allow compute instances to have permissions
\end{itemize}

% 10. Using Java sdk to interact with OCI?
\section{Using Java SDK to interact with OCI?}
OCI provides a Java \acrshort{sdk} to programmatically interact with its services. To use it:

\begin{enumerate}
  \item {
      Step 1: Add OCI SDK Dependency Maven
      \begin{lstlisting}[language=xml]
        <dependency>
        <groupId>com.oracle.oci.sdk</groupId>
        <artifactId>oci-java-sdk-common</artifactId>
        <version>3.0.0</version>
        </dependency>
    \end{lstlisting}
  }
\item {
    Step 2: Configure Authentication
    Use an \acrshort{api} key or Instance Principals for authentication.
  }
\item {
    Step 3: Initialize the SDK \& Make a Request
      \begin{lstlisting}[language=Java, caption={lists all OCI regions using SDK}]
        import com.oracle.bmc.auth.ConfigFileAuthenticationDetailsProvider;
        import com.oracle.bmc.identity.IdentityClient;
        import com.oracle.bmc.identity.requests.ListRegionsRequest;
        import com.oracle.bmc.identity.responses.ListRegionsResponse;

        public class OCIExample {
          public static void main(String[] args) throws Exception {
            ConfigFileAuthenticationDetailsProvider provider = new ConfigFileAuthenticationDetailsProvider("~/.oci/config");
            IdentityClient identityClient = new IdentityClient(provider);
            ListRegionsResponse response = identityClient.listRegions(ListRegionsRequest.builder().build());
            response.getItems().forEach(region -> System.out.println(region.getName()));
            identityClient.close();
          }
        }
      \end{lstlisting}
      \begin{lstlisting}[language=Java, caption={lists all compute instances in a specified compartment}]
        import com.oracle.bmc.auth.ConfigFileAuthenticationDetailsProvider;
        import com.oracle.bmc.core.ComputeClient;
        import com.oracle.bmc.core.requests.ListInstancesRequest;
        import com.oracle.bmc.core.responses.ListInstancesResponse;

        public class OCITest {
          public static void main(String[] args) throws Exception {
            ConfigFileAuthenticationDetailsProvider provider = new ConfigFileAuthenticationDetailsProvider("~/.oci/config", "DEFAULT");
            ComputeClient computeClient = new ComputeClient(provider);
            ListInstancesRequest request = ListInstancesRequest.builder()
            .compartmentId("ocid1.compartment.oc1..example")
            .build();
            ListInstancesResponse response = computeClient.listInstances(request);
            response.getItems().forEach(instance -> System.out.println(instance.getDisplayName()));
            computeClient.close();
          }
        }
      \end{lstlisting}
  }
\end{enumerate}

\section{What is CIDR?}
\acrfull{cidr} is a method used to allocate \acrshort{ip} addresses and route internet traffic efficiently. 
It replaces the traditional class based \acrshort{ip} addressing system (class A, B, C, D) with a more flexible, hierarchical, and scalable system.
\subsection{CIDR Notation}
CIDR notation represents an IP address range using a prefix and a subnet mask length, written as \lstinline{IP Address / Subnet Mask Length}
\\
Example:
\\
$192.168.1.0/24$
\begin{itemize}
  \item $192.168.1.0 \rightarrow$ Network address
  \item $/24$ $\rightarrow$ The first 24 bits are the network portion, leaving 8 bits for host addresses.
\end{itemize}
This means:
\begin{itemize}
  \item The subnet can have 256 IPs (from $192.168.1.0$ to $192.168.1.255$).
  \item Usable IPs: $254$ (since the first IP is the network address and the last is the broadcast address).
\end{itemize}

CIDR Subnet Mask Table
\begin{center}
  \begin{tabular}{ ||c|c|c|c|| }
    \hline
    CIDR & Subnet Mask & \# of Addresses & \# of usable IPs \\ [0.5ex]
    \hline \hline
    $/8$ & $255.0.0.0$ & $16,777,216$ & $16,777,214$ \\
    \hline
    $/16$ & $255.255.0.0$ & $65,536$ & $65,534$ \\
    \hline
    $/24$ & $255.255.255.0$ & $256$ & $254$ \\
    \hline
    $/30$ & $255.255.255.252$ & $4$ & $2$ \\
    \hline
  \end{tabular}
\end{center}
\subsection{CIDR Benefits}
\begin{enumerate}
  \item Efficient IP Address Allocation $\rightarrow$ Avoids wasteful allocation of large blocks.
  \item Better Routing $\rightarrow$ Reduces the size of routing tables by aggregating multiple IPs (supernetting).
  \item Flexible Subnetting $\rightarrow$ Allows customized network sizes without being restricted to Class A, B, or C.
  \item Supports \acrfull{vlsm} $\rightarrow$ Enables different subnet sizes within the same network.
\end{enumerate}

\subsection{Example Use Case}
\begin{enumerate}
  \item Public Cloud Networking (\acrfull{aws}, \acrfull{oci}, Azure, \acrfull{gcp}) $\rightarrow$ \acrshort{cidr} blocks are used to define \acrfull{vpc} address ranges.
  \item Corporate Networks $\rightarrow$ Organizations define \acrshort{cidr} blocks to segment their internal IP addressing.
\end{enumerate}

\section{Traditional IP Address Classes (Before CIDR)}
\begin{center}
  \resizebox{\textwidth}{!}{
    \begin{tabular}{ ||c|c|c|c|c|c|| }
      \hline
      Class & Starting IP & Ending IP & Default Subnet Mask & \# of Networks & \# of Hosts per Network \\ [0.5ex]
      \hline \hline
      A & 1.0.0.0 & 126.255.255.255 & 255.0.0.0 & 128 (0 \& 127 reserved) & 16,777,214 \\
      \hline
      B & 128.0.0.0 & 191.255.255.255 & 255.255.0.0 & 16,384 & 65,534 \\
      \hline
      C & 192.0.0.0 & 223.255.255.255 & 255.255.255.0 & 2,097,152 & 254 \\
      \hline
      D (Multicast) & 224.0.0.0 & 239.255.255.255 & N/A & N/A & N/A \\
      \hline
      E (Reserved) & 240.0.0.0 & 255.255.255.255 & N/A & N/A & N/A \\
      \hline
    \end{tabular}
  }
\end{center}


\subsection{Problems with Class-based IP Addressing}
\begin{enumerate}
  \item Wasted IPs: Organizations were forced to use fixed-size blocks (e.g., Class B had 65,534 hosts, even if only 500 were needed).
  \item Limited Growth: The rigid structure led to a shortage of IP addresses.
\end{enumerate}
\subsection{CIDR and Its Elimination of Classes}
CIDR eliminates fixed class sizes and allows arbitrary subnet sizes using the "/prefix notation", making IP allocation more efficient.
\subsection{CIDR Addressing Example}
Instead of using Class A, B, or C, CIDR uses prefix lengths (e.g., /16, /24, /26).
Example:\\
192.168.1.0/26 (instead of Class C /24)\\
Subnet Mask: 255.255.255.192\\
Available IPs: 64 (2 reserved: network \& broadcast)\\
Usable IPs: 62\\
This flexibility allows ISPs and enterprises to allocate only the required number of IPs, avoiding unnecessary waste.

\section{Control Plane vs. Data Plane}
In networking and cloud computing, the Control Plane and Data Plane are two fundamental components that define how network traffic is managed and processed.
\subsection{Control Plane}
\subsection{Data Plane}

\section{User Principal}
\section{Instance Principal}
\section{Resource Principal}
\section{IP and Subnet in OCI}

% \printglossaries
\end{document}
