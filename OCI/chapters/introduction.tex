\documentclass[../main.tex]{subfiles}

% \begin{lstlisting}[language=GraphQL, caption={Employee fetch id and name only using GraphQL query}, label={lst:fetch-employee-id-first-name}]
% {
%   employees {
%     id
%     firstName
%   }
% }
% \end{lstlisting}
% Listing  \ref{lst:fetch-employee-id-first-name} is a basic example of usage.

\graphicspath{{\subfix{../images/}}}
\begin{document}
\chapter{Introduction}
% 01. What is OCI?
\section{What is OCI?}
\gls{OCI} is a cloud computing service offered by Oracle that provides \gls{IaaS}, \gls{PaaS}, \gls{SaaS}, and \gls{DaaS} solutions.
It is designed to support high-performance computing, enterprise workloads, and cloud-native applications.
OCI provides services such as compute, storage, networking, databases, security, and \gls{AI}/\gls{ML} tools.
% 02. What is region in OCI?
% 03. What is realm in OCI?
% 04. What is physical domain and fault domain in OCI?
% 05. What is tenancy in OCI?
% 06. What is compartment in OCI?
% 07. What is group and dynamic group in OCI?
% 08. How is auth done in OCI?
% 09. Understanding IAM in OCI?
% 10. Using Java sdk to interact with OCI?

\printglossaries
\end{document}
