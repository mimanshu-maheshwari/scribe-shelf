\documentclass[../main.tex]{subfiles}

\graphicspath{{\subfix{../images/}}}
\begin{document}

\chapter{Connection to MongoDB Database}
\section{Using MongoDB Connection Strings}
Connection string can be used to connect from MongoShell, MongoDB Compass, or any other application.
It provides standered fromat and DNS seed list format.
\subsubsection{Standered Format}
Used to connect to standalone clusters, replica sets, or sharded clusters.
\subsubsection{DNS Seed List Format}
Released in MongoDB 3.6.
\subsection{Connection String}
It allows to provide a DNS server list to connection string. Gives more flexibility of deployment.
Ability to change servers in rotation without reconfiguring clients.
\newline
In atlas it is present in Database tab $\rightarrow$ Connect $\rightarrow$ Connect to application.
\newline
Connection String: 
\lstinline{mongodb+srv://<username>:<password>@cluster0.usqsf.mongodb.net/?retryWrites=true&w=majority}
If port number is not specified MongoDB will default to $27017$.

\section{Connecting to a MongoDB Atlas Cluster with the Shell}
mongsh: MongoDB shell is a nodejs \gls{REPL} environment.
It gives us access to JavaScript variables, functions, conditionals, loops and Control flow statements inside shell.
\begin{lstlisting}[caption=shell example,language=TypeScript] 
const greetingArray = ["hello", "world", "welcome"]; 
const loopArray = (array) => array.forEach(el => console.log(el));
loopArray(greetingArray);
\end{lstlisting}
In atlas it is present in Database tab $\rightarrow$ Connect $\rightarrow$ Connect with the MongoDB Shell.
\begin{lstlisting}[caption=connect to db,language=bash] 
atlas clusters connectionStrings describe myAtlasClusterEDU
MY_ATLAS_CONNECTION_STRING=$(atlas clusters connectionStrings describe myAtlasClusterEDU | sed "1 d")
mongosh -u myAtlasDBUser -p myatlas-001 &MY_ATLAS_CONNECTION_STRING
\end{lstlisting}

\section{Connecting to a MongoDB Atlas Cluster with Compass}
In atlas it is present in Database tab $\rightarrow$ Connect $\rightarrow$ Connect using MongoDB Compass.

\section{Connecting to a MongoDB Atlas Cluster from an Application}
\subsubsection{MongoDB drivers} 
Connect our application to our database by using a connect string.
You can find the \href{mongodb.com/docs/drivers}{\color{blue}{mongodb drivers list here}}.

\section{Troubleshooting MongoDB Atlas Connection Errors}


\printglossaries

\end{document}

