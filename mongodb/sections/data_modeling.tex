\documentclass[../main.tex]{subfiles}

\graphicspath{{\subfix{../images/}}}
\begin{document}

\chapter{Data Modeling}
\section{Introduction to Data Modeling}
Data Modeling is process of defining how data is stored and the relationships and the relationships that exists among different entities in your data. \\ 
We refer to Organization of data inside a database as a \textbf{Schema}. \\ 
To develop schema think about your database and you need to ask these questions:
\begin{itemize}
	\item{What does my application do?}
	\item{What data will I store?}
	\item{How will users access this data?}
	\item{What data will be most valuable to me?}
\end{itemize}

Asking these questions will help you to:
\begin{itemize}
	\item{descibe your task as well those of users. }
	\item{What you data looks like?}
	\item{the relationships among the data.}
	\item{The tooling you plan to have.}
	\item{The access patterns that might emerge.}
\end{itemize}

\medskip
A good data model can: 
\begin{itemize}
	\item{Make it easier to manage data}
	\item{Make queries more efficient}
	\item{Use less memory and \gls{CPU}}
	\item{Reduce cost}
\end{itemize}

As a general pricipal of MongoDB data that is accessed together should be stored together. \\ 
Collections do not enforce any document structure by default. That means document even those in the same collection can have different structures. \\
Each document at MongoDB can be different which is known as \textbf{Polymorphism}. \\ 
Embedded document model enables us to build complex relationships among data. You can normalize your data by using database references. \\ 
The key pricipal here is how your application will use the data rather than how it's stored in the database.



\section{Types of Data Relationships}
\subsection{Different types of relationships data can have: }
\begin{itemize}
	\item{One-to-one}
	\item{One-to-many}
	\item{Many-to-Many}
\end{itemize}

\subsubsection{One-to-One}
A relationship where a data entity in one set is connected to exactly one data entity in another set.

\subsubsection{One-to-Many}
A relationship where a data entity in one set is connected to any number of entities in another set.
\subsubsection{Many-to-Many}
A relationship where any number of data entities in one set are connected to any number of data entities in another set.


\subsection{Two main ways to model these relationships: }
\begin{itemize}
	\item{Embeding}
	\item{Referencing}
\end{itemize}
\subsubsection{Embeding}
We take related data and insert it into our document.
\subsubsection{Referencing}
When we refer to documents in another collection in our document. Reference is made by their respective \lstinline{ObjectID}.

\section{Modeling Data Relationships}
\subsection{Embedding Data in Documents}
Also known as nested document.
\begin{itemize}
	\item Avoids application joins. 
	\item Provides better performance for read operatinos.
	\item Allows developers to update related data in a single write operations.
\end{itemize}
\subsubsection {Issues:}

\begin{itemize}
	\item Embedding data into a single document can create large documents.
	\item{Large documents have to be read into memory in full, which can result in a slow application performance for you end user.}
	\item{Continuesly adding data without limit creates \textbf{unbounded documents.}}
	\item{Unbounded Documents may exceed the \gls{BSON} document threshold of 16MB. This is schema anit-patterns which you should avoid.}
\end{itemize}

\subsection{Referencing Data in Documents}
\printglossaries
\end{document}
