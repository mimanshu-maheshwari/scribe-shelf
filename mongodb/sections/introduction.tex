\documentclass[../main.tex]{subfiles}

\graphicspath{{\subfix{../images/}}}
\begin{document}

\chapter{Introduction}
\section{Getting Started with MongoDB Atlas}
\subsection{Creating and Deploying on Atlas Cluster}
\begin{enumerate}
	\item{Create a Organization}
	\item{Create a Project under Organization}
	\item{Create a cluster $\rightarrow$ Advanced Settings $\rightarrow$ Shared M0 cluster $\rightarrow$ Add username and password $\rightarrow$ Add you ip as allowed origin.}
	\item{Open Database tab and navigate to your cluster}
	\item{Load default data}
	\item{Navigate to collections tab to see loaded clusters}
	\item{Sample database fetch from analytics accounts table
		\begin{lstlisting}[caption=account test data,language=json]
{
	"_id": {
		"$oid": "5ca4bbc7a2dd94ee5816238c"
	},
	"account_id": {
		"$numberInt": "371138"
	},
	"limit": {
		"$numberInt": "9000"
	},
	"products": [
		"Derivatives",
		"InvestmentStock"
	]
}
		\end{lstlisting}
	}
\item{To filter on account id:
		\begin{lstlisting}[caption=filter on account id,language=json]
{
	account_id: 794875
}
	\end{lstlisting}
}
\item{To login into atlas login use \lstinline[language=bash,caption=atlas authentication]{atlas auth login}}
\item{create cluster with user and password:
		\begin{lstlisting}[caption=load cluster,language=bash]
atlas setup --clusterName myAtlasClusterEDU --provider WS --currentIp --skipSampleData --username myAtlasDBUser --password myatlas-001 --projectId 66d19c1f26ef8b512df3b41 | tee atlas_cluster_details.txt
		\end{lstlisting}
	}
\item{Load sample data: \lstinline[caption=load sample data, language=bash]{atlas clusters smapleData load myAtlasClusterEDU} }

\end{enumerate}

\section{Introduction to MongoDB}
\subsection{Key Terms}
\subsubsection{Document}
The basic using of data in MongoDB
\subsubsection{Collections}
A grouping of documents. Documents in collection are typically similar they don't necessarily have same structure.
\subsubsection{Database}
A container for Collections

\section{MongoDB and The Document Model}
\subsection{The MongoDB Document Model}
\begin{itemize}
	\item{MongoDB stores data in structures known as documents}
\end{itemize}

\subsubsection{Document}
Docuement are displayed in \gls{JSON} format but stored in \gls{BSON} format.
\gls{BSON} also supports additional data types that are unavailable in \gls{JSON}

It can support All \gls{JSON} data types as well as Dates, Numbers, Object Id's, and more!
 
ObjectID is used to create unique identifires.
Every Document requires an $_id$ field, which acts as a primary key.
If an inserted document doesn't include the $_id$ field, MongoDB automatically generates an ObjectId for the $_id$ field.
MongoDB supports flexible schema model and polymorphic data, this allows us to store documents with different structure together.
Documents may contain fields. Fields may contain different types.

\subsubsection{Optional Schema validation}
Set of constraints on the structure of documents.

\section{Managing Databases, Colections, and Documents in Atlas Data Explorer}
Explore Atlas


\printglossaries
\end{document}
