\documentclass[../main.tex]{subfiles}

\graphicspath{{\subfix{../images/}}}
\begin{document}

\chapter{MongoDB CRUD Operations: Insert and Find Documents}
\section{Inserting Documents in MongoDB Collections}
Their are two methods 
\lstinline {insertOne()} and \lstinline {insertMany()}.
\newline
Usage: \lstinline {db.<collection>.insertOn()}.
If the collection doesn't exist then MongoDB creates the collection when we use \lstinline{insertOne()}.
Eg: 
\begin{lstlisting}[caption=insertOne using mongosh, language=JavaScript]
	db.grades.insertOne({
		student_id: 546799, 
		scores: [
		{
			type: "quiz", 
			score: 50,
		},
		{
			type: "homework", 
			score: 70,
		}
		]
	})
\end{lstlisting}

\begin{lstlisting}[caption=output of insertOne using mongosh, language=JSON]
	{
		acknowledged: true,
		insertedId: ObjectId("...")
	}
\end{lstlisting}

\begin{lstlisting}[caption=insertMany using mongosh, language=JavaScript]
	db.grades.insertMany([
	{
		student_id: 546799, 
		scores: [
		{
			type: "quiz", 
			score: 50,
		},
		{
			type: "homework", 
			score: 70,
		}
		]
	},
	{
		student_id: 546800, 
		scores: [
		{
			type: "quiz", 
			score: 50,
		},
		{
			type: "homework", 
			score: 70,
		}
		]
	}
	])
\end{lstlisting}
\begin{lstlisting}[caption=output of insertMany using mongosh, language=JSON]
	{
		acknowledged: true,
		insertedIds: [
		'0': ObjectId("..."),
		'1': ObjectId("..."),
		]
	}
\end{lstlisting}

\section{Finding documents in MongoDB Collections}
Usage: \lstinline{db.<collection>.find()}
\begin{lstlisting}[caption=find method in mongosh, language=JavaScript]
	use training; // database 
	db.zips.find(); // zips are collection 
	// use "it" for iterating from more result
\end{lstlisting}

\subsection{eq operator}
\begin{lstlisting}[language=MongoDB]
	{ field: { $eq: <value> } }
	# or
	{ field: <value> }
\end{lstlisting}

\subsection{in operator}
\begin{lstlisting}[language=MongoDB]
	{ field: { $in: [<value1>, <value2>, <value3>,..] } }
\end{lstlisting}

\section{Finding Documents by using Comparison Operators}
\subsection{Usage:}
\begin{lstlisting}[language=MongoDB,caption=Comparison Operators]
	{ field: { <operator>: <value> } }
\end{lstlisting}
\subsection{gt: greater than}
\begin{lstlisting}[language=MongoDB, caption=greater than gt]
	{ field: { $gt : <value> } }
\end{lstlisting}
\subsection{lt: less than}
\begin{lstlisting}[language=MongoDB, caption=less than lt]
	{ field: { $lt : <value> } }
\end{lstlisting}
\subsection{gte: greater than or equal}
\begin{lstlisting}[language=MongoDB, caption= greater than equal gte]
	{ field: { $gte : <value> } }
\end{lstlisting}
\subsection{lte: less than or equal}
\begin{lstlisting}[language=MongoDB, caption=less than equal lte]
	{ field: { $lte: <value> } }
\end{lstlisting}

\section{Querying on Array elements in MongoDB}
\begin{lstlisting}[language=MongoDB]
	db.accounts.find({"products": "InvestmentStock"})
\end{lstlisting}

\lstinline{$elemMatch} ensures product field is an array.

\begin{lstlisting}[language=MongoDB, caption=elemMatch operator]
	db.accounts.find({
		products: {
			$elemMatch: {
				<query1>,
				<query2>,
				<query3>,
				...
			}
		}
	})
\end{lstlisting}


\begin{lstlisting}[language=MongoDB]
	db.accounts.find({
		products: {
			$elemMatch: { $eq: "InvestmentStock" }
		}
	})
\end{lstlisting}


\section{Finding Documents by Using Logical Operators}
\subsection{and operator}
Explicit syntax:
\begin{lstlisting}[language=MongoDB,caption=and operator explicit]
	db.accounts.find({
		$and: [
		{<expression>},
		{<expression>},
		...
		]
	})
\end{lstlisting}

Implicit syntax:
\begin{lstlisting}[language=MongoDB, caption=and operator implicit]
	db.accounts.find({ {<expression>}, {<expression>} })
\end{lstlisting}

\subsection{and operator}
\begin{lstlisting}[language=MongoDB,caption=or operator]
	db.accounts.find({
		$or: [
		{<expression>},
		{<expression>},
		...
		]
	})
\end{lstlisting}

\subsection{Combining OR and AND}
Both operators can be combined. 
But we can not have two fields with same name in same \gls{JSON} object.
So if we try implicit and with two or queries the first or will be overwritten by second or query.
\begin{lstlisting}[language=MongoDB]
	db.accounts.find({
		$and: [
		{ $or: [
			{<expression>},
			{<expression>},
			...
		]},
		{ $or: [
			{<expression>},
			{<expression>},
			...
		]}
		]
	})
\end{lstlisting}
But if we write this in implicit and way: 

\begin{lstlisting}[language=MongoDB]
	db.accounts.find({
		$or: [ {<expression>}, {<expression>}, ... ],
		$or: [ {<expression>}, {<expression>}, ... ]
	})
\end{lstlisting}
It will not give same output as second or will overwrite first or.


\printglossaries

\end{document}

